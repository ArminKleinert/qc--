
% vim: ts=8 sw=4 et:

\documentclass[11pt]{article}
\usepackage{path}

\author{Christian Lindig}
\title{Roadmap to a Working Quick C-{}- Compiler}

\newcommand\PAL{{\small C-{}-}}
\newcommand\AST{{\small AST}}
\newcommand\CFG{{\small CFG}}
\newcommand\qcc{{\small QC-{}-}}
\newcommand\rtl{{\small RTL}}
\newcommand\burg{{\small BURG}}
\newcommand\cfg{{\small CFG}}
\newcommand\ocaml{{\small OCAML}}

\parindent0pt
\parskip1.5ex

% ------------------------------------------------------------------ 
\begin{document}
% ------------------------------------------------------------------ 

\maketitle

As pointed out in \textit{Roadmap to a Better Quick C-{}- Compiler}, the
Quick C-{}- compiler suffers from some design flaws. Aside from that, it
has holes that prevent the compiler compile all legal {\PAL} programs.
Below is a list of holes in need of a fix, from higher to lower
priorities.  

The main overall problem is complexity. The development of the compiler
will benefit from trying to cut down the number of modules,
abstractions, and moving parts in general. 

Telling from the time it took to get certain parts working at all, the
following points are notorious for trouble and should be considered for
a re-design. 

\begin{itemize}

\item The control-flow graph. Details are outlined in \textit{Roadmap to
a Better Quick C-{}- Compiler}.

\item Integration of machine generated code. Basically it does not work
because machine generated code is not flexible enough to handle
temporaries and late compile-time constants. Adjusting the machine
generated code manually is time-consuming and error prone. Machine
generated code provides {\rtl} constructors for machine instructions, a
recognizer for machine instructions in {\rtl} form, and an emitter of
assembly code.

\item Relocatable addresses have undergone several changes in their
representation.  There are still two slightly different representations
used in the compiler that wait to be unified: \path|Reloc| and one in
\path|Sledlib|.  

\item The semantics of the assembler interface needs to be defined
clearly. The problem stems for an anticipated use for both textual
assemblers and binary emitters. Especially the role of symbols is not
clear and should be explained with respect to the better known textual
assemblers.

\end{itemize}

There is some confusion in the way {\PAL} primitive operators are
translated to {\rtl} operators. {\PAL} operators are monomorphic in
their result type, whereas {\rtl} operators are not. A polymorphic
{\rtl} operator takes an additional argument compared to its monomorphic
{\PAL} counter part. This argument should be supplied first to take
advantage of currying: the a polymorphic {\rtl} operator supplied with
its first argument implements the monomorphic {\PAL} operator. Currently
the additional argument is the last argument. \textit{I hope this
problem is solved in \path|Rtlop|. The above explanation of operator
specialization is not entirely correct. The specializing parameter is
not a formal argument of a primitive, but part of its \emph{width list}.
--CL}

According to the {\PAL} manual, integer multiplication \path|%mul| is
quasi-polymorphic in its result and thus must be used as in \path|%mul32|.
This poses a problem for the translation of the infix operator
``\path|*|''.  What should \path|*| stand for?  Currently, \path|%mul|
is treated as monomorphic in its result size. \textit{Norman has
clarified this: \path|*| stands for a monomorphic multiplications, the
general multiplication is still quasi-polymorphic. Internally this is
now implemented with two operators: \path|%mul| and \path|%mul@|. Any
better idea? --CL (Tue Apr 30 13:15:48 EDT 2002)}

The algorithm that linearizes the control-flow graph does not work
properly. It fails to detect the difference between nodes that fall into
a label and those that jump to a label. A probable fix to the problem is
always to jump to a label and never to fall into it. When the design of
the {\cfg} is reconsidered, this problem should be addressed, too.

Targets provide code that spills and reloads registers. This code must
respect the machine invariant, but currently does not. Machine specific
spilling and reloading code must be defined. This code might be tied to
the question how small values of 8 or 16 bit widths are handled. Are
these values ever spilled?

The compiler lacks a comprehensive plan how to deal with values of
smaller than the size of a register. 

A calling conventions needs to be defined and implemented. The currently
implemented calling convention is a toy and not useful for real calls.

After an {\rtl} has left the register allocator, it may still contain
late compile-time constants. After these are substituted by bit vectors,
the {\rtl} must undergo a constant folding process to make sure it
respects the machine invariant. Such a constant folding routine
currently does not exist.  Since the front end also needs to evaluate
compile-time and link-time {\rtl} expression, one general {\rtl}
evaluation mechanism would be appropriate. The current module
\path|rtleval| can only handle compile-time and link-time {\rtl}s. 
\textit{I hope this problem is solved in \path|Rtleval2|. --CL}

{\rtl} operators describe compile-time, link-time, and run-time
operations. The list of operators available at compile-time is extremely
limited. Some implementation are just dummies that return constant
results. 

A target description provides {\rtl}s for instructions that change the
control flow, like \path|jump|, \path|branch|, or \path|return|.  These
{\rtl}s have the strange property that they are target specific on one
hand, but are not representable as a single machine instruction on the
other. The reason is, that they include complex expressions. The
invariant of these {\rtl}s should be clearly defined.

The floating-point comparison \path|fcmp| returns two bits to indicate
the result. The semantics of the four possible values need to be
defined. (While NR believes that the IEEE 754 standard defines them, CL
doubts it.)

The {\small SPARC} cannot move a value from an integer register to a
floating-point register. The Code expander that wants to implement such
moves needs a memory cell as temporary storage. What is the address of
such a cell. In principle, every thread only needs one such cell. Should
the address be on the stack, expressed using  a late compile-time
constant? How do we reserve such a slot \emph{in the code expander}?

Currently the overall stack layout is hardcoded into the \path|freeze|
function in the \path|Ast3IR| module. This seems a bad idea, because
this module is supposed to be target independent and thus defines the
frame layout for all targets. A more flexible architecture is needed.

Support for floating-point numbers is basically absent. The main reason
is that floating-point numbers and integers have the same (bit vector)
type in {\PAL}, but different types in {\ocaml}. We lack support to read
a floating point number from a string into bit vector and compile-time
floating point operations on bit vectors. In the \path|rtl/| directory
the C~code \path|unit64p.c| provides casting of 64 bit floating point
values to 64 bit integer values. However, this code is platform
sensitive and maybe the wrong way.

Continuation are poorly supported. In particular, the control-flow graph
lacks proper edges to indicate the data flow in the presence of
continuations. Some continuations cannot escape a given procedure and
thus do not require a run-time representation on the stack. An escape
analysis is missing to find non-escaping continuations.

The implementation of \path|cutto| currently lacks all details to find
the activation for the continuation and thus the \emph{incoming} area in
that activations. This is where the \path|cut to| statement puts the
\emph{outgoing} overflow parameters. Some abstraction for a
continuation's run-time representation, similar to \path|Automaton.loc|
might be appropriate.

Neither the front end, nor the back end can handle literal strings. The
feature was delayed over a discussion how to handle Unicode characters.
The changes proposed by NR to the {\PAL} language definition might be
relevant here.

The front end is unaware of a chosen back end. This precludes it from
rejecting certain {\PAL} programs that do type check, but that the back
end cannot compile. Things to front end does not know: supported with of
registers, supported width of operators, names of calling conventions,
and hardware registers.


The {\PAL} language construct \path|span| is unsupported.

The {\PAL} language construct \path|switch| is unsupported.

When multiple compilation units share their global variables one of them
must define them, while the other import them on an assembler level. No
mechanism exists to mark the ``main'' compilation unit that defines all
global variables.

Expression evaluation needs to be unified. The correct way is to
translate an expression into an {\rtl} expressions and then to evaluate
it. Currently the constant evaluation phase of the \path|Elab| module
evaluates constant expressions directly. See also the reasons for this
in module \path|Elab.Const.Eval|:

\begin{quote}\it
    Several reasons: We would need a translation from {\AST} to
    \path|Rtl.env|. The existing one in \path|ast3ir| is written against
    the \path|Fenv.Clean| interface, but we need one written against the
    \path|Fenv.Dirty| interface. The translation of an application needs
    the widths of its arguments. Therefore type-checking using
    \path|Expcheck| would be done twice: once before translation and then
    again during translation. Detailed error messages are simpler to
    generate with the existing approach. 
\end{quote}

Export substitution of late compile-time constants and constant folding
as finalizing step of a procedure translation to Lua. Currently these
steps are hard-coded into the \path|freeze| function in module
\path|Ast3IR|.


% ------------------------------------------------------------------ 
\end{document}
% ------------------------------------------------------------------ 

