
% vim: ts=8 sw=4 et:

\documentclass[11pt]{article}
\usepackage{path}

\author{Christian Lindig\\ Norman Ramsey}
\title{Roadmap to a Working Quick C-{}- Compiler}
\date{Early June, 2002}

\newcommand\PAL{{\small C-{}-}}
\newcommand\AST{{\small AST}}
\newcommand\CFG{{\small CFG}}
\newcommand\qcc{{\small QC-{}-}}
\newcommand\rtl{{\small RTL}}
\newcommand\burg{{\small BURG}}
\newcommand\cfg{{\small CFG}}
\newcommand\ocaml{{\small OCAML}}

\newdimen\whoskip
\whoskip=2.5em
\newcommand\who[1]{{\unskip\nobreak\hfil\penalty50
    \hskip\whoskip\hbox{}\nobreak\hfil\textsl{#1}\parfillskip=0pt
    \finalhyphendemerits=0 \par}}

\parindent0pt
\parskip1.5ex

\renewcommand\b[1]{\leavevmode\llap{\rlap{[$\mathbf{#1}$]}\hskip 6em}}

% ------------------------------------------------------------------ 
\begin{document}
% ------------------------------------------------------------------ 

\maketitle

The purpose of this document is to explain what needs to be done by
whom to lead to a working Quick~{\PAL} compiler.
This document emphasize pieces that are missing, broken, or
incomplete.
Pieces that need refactoring are covered in the 
\textit{Roadmap to a Better Quick C-{}- Compiler}.


\begin{enumerate}
\item\b{A}
The control-flow graph should be refactored, as  outlined in \textit{Roadmap to
a Better Quick C-{}- Compiler}.
This is a major item to be covered by the ``control-flow graph working
group.'' 
In addition to the problems listed in that other document, the working
group should also address the following problems.
\begin{enumerate} 
\item
At least one operation in the current implementation of the {\cfg}
requires the introduction of a new label together with a new symbol for
that label. Symbols are issued by an assembler, but this context is
missing. As a hack, a symbol is created locally. However, this is wrong
because it does not reflect the rules of the actual assembler. I would
like to see a design where new branches can be introduced without
worries about target specific symbols. This plays into the observation,
that {\PAL} has really two classes of labels: local labels introduced by
the translation of statements like \path|if|, and global labels that are
define in a {\PAL} source file. At least local branches should not
depend on symbols.
\item
When the control-flow graph changes, what clients should be re-done
first?
Picking the right clients means we get good early feedback on the
redesign.
Possible clients might include \texttt{ast2ir}, a register allocator,
and a code expander (especially for short-circuit Booleans).
\end{enumerate}
\who{Christian, Glenn, John, Kevin, and Norman.}
\item\b{B+}
\emph{Identify suitable technology for writing machine-dependent
components}.  
 Generating everything from a machine is a research
problem, and the solutions are not ready for use in the compiler.
We must identify techniques that will produce solid code, which
we expect to be primarily hand-written code.
\who{Norman.}
\item\b{B+}
Build a \emph{back end for a new target}, developing detailed porting
instructions on the way.
This process will include solutions to the following problems:
\begin{enumerate}
\item 
A target description provides {\rtl}s for instructions that change the
control flow, like \path|jump|, \path|branch|, or \path|return|.  These
{\rtl}s have the strange property that they are target specific on one
hand, but are not representable as a single machine instruction on the
other. The reason is, that they include complex expressions. The
invariant of these {\rtl}s should be clearly defined.

\emph{The solution is likely to involve reading the PC from a
conventional location, writing a PC to a conventional location, and
possibly a private invariant shared with the code expander. ---NR}
\item
How does the optimizer know if it is safe to negate the condition on a
conditional branch without violating the machine invariant?
And if it not safe, what can the optimizer do?
\item
Figure out how to \emph{reserve a stack slot in the code expander}.
The {\small SPARC} cannot move a value from an integer register to a
floating-point register. 
The code expander, which must implement such a move, needs a memory
cell as temporary storage. 
What is the address of this cell?
In principle, we need only one such cell per program.  (Or one per
thread, if scheduling is pre-emptive.)
We might wish to allocate one per activation, in the stack slot of
the current frame, because getting to a cell on the stack is faster
and doesn't cost a register.
\item
\emph{Stack-frame layout} should be part of the target specification.
Currently the overall stack layout is hardcoded into the \path|freeze|
function in the \path|Ast3IR| module. 
\end{enumerate}
  \who{Norman.}
\item\b{B}
Write \emph{variable placers} by hand.  \who{Kevin.}
\item\b{B}
Develop either hand-written or machine-generated
\emph{operator context charts} to guide selection of temporaries.
For use both in variable placement and in code expansion.
\who{Kevin.}
\item\b{D}
{\rtl} operators describe compile-time, link-time, and run-time
operations. The list of \emph{operators available at compile time} is extremely
limited.
Someone should analyze the situation and work on the following items.
\begin{itemize}
\item
We need precise documentation (perhaps even formal, via Lua?) of what
 operators are supported at compile time.
N.B. an ``operator'' means a fully instantiated operator at a
 particular type, not a polymorphic promise that can be redeemed at
 any type.
\item
We need actual implementations of the compile-time operators.
\item
 Some current implementations are dummies that return constant
results; these implementations must be removed.
\end{itemize}
\who{Kevin.}
\item\b{A}
Identify the \emph{critical compile-time operators} and make sure they
are implemented correctly.
These might include add and subtract, bit insert, bit extract, and
shifts.
\who{Christian.}
\item\b C
Partial \emph{map of variables and callee-saves registers to locations} at
each call site.
\who{John or Kevin.}
\item\b C 
Implement \texttt{span}, with support for lookup at each call site.
\who{John or Kevin.}
\item \b B
Specify and implement \emph{calling conventions.}
Christian is quite interested in this problem.
\who{John or Christian.}
\item \b A
Develop and successfully compile a small \emph{test suite} for SPARC,
including some C~programs.
Because of limitations of the compiler, it is likely that any
C~programs we can successfully compile and run will be carefully
crafted for this purpose.
It is OK if these programs don't do anything useful.
\who{Christian.}
\item \b{B+}
Implement \emph{small arithmetic}. 
The compiler currently lacks a comprehensive plan how to deal with values of
smaller than the size of a register. 
\who{Kevin and Norman.}
\item \b A
Support a \emph{complete set of RTL operators}, probably with new
representation.
Minimum here is to make sure that we have all 60+ operators and that
everything typechecks. 
A new representation could be a bonus.
\who{Kevin.}
\item\b{A-}
Identify, allocate, and initialize escaping \emph{continuations}.
This entails repairing the following problems:
\begin{enumerate} 
\item
Document the representation of a continuation, including space set
aside for incoming overflow parameters.
Some internal abstraction for a
continuation's run-time representation, similar to \path|Automaton.loc|
might be appropriate.
\item
Identify which continuations escape.
\item
The translation of continuations in \path|Ast3ir| is bogus. A
continuation \emph{value} is a pointer into the stack where two values
are stored: a pointer to the code of the continuation, and a stack
pointer. The current representation of a continuation value is a
link-time constant, which is wrong. A continuation \emph{value} is a
late compile-time expression: a sum of the stack pointer value and some
late compile-time constant. The stack pointer in turn, depends on the
currently active calling convention. \emph{I have introduced a new
abstraction \path|Contn| that addresses some of the problems. However,
most things are not yet integrated and the translation of expressions
with included continuation values is still wrong. -- CL}
\end{enumerate}
Here are some notes on this problem:
\begin{itemize}
\item
The target must hide the representation of a continuation.
\item
The target will provide these values:
\begin{itemize}
\item
A block for continuations
\item
Function from parameter list to block (or size/alignment) for a
particular continuation
\item
Function from stack location, code label to RTLs that initialize a
continuation
\item
An implementation of \texttt{cut to}.
\end{itemize}
The front end will
\begin{enumerate}
\item
Find which continuations escape
\item
Allocate space for each one, managing the name of a continuation
exactly as we manage a label in \texttt{stackdata}
\item
Emit initialization code obtained from the target
\end{enumerate}
\end{itemize}
\who{Christian.}
\item\b{B+}
Implement \texttt{cut to} correctly.
The implementation of \path|cut to| currently expects to put overflow
parameters into the  \emph{incoming} area for the procedure containing
the continuation.
This is a bad plan; it is better for each continuation to have its own
overflow area reserved on the stack.
We believe this strategy will perform acceptably well because we
expect most continuations to take at most one or two parameters, which
can easily be passed in registers. 
(Consider that any callee-saves registers is fair game for a
continuation parameter.)
\who{Christian.}
\item \b{B}
Solve the following problem, which NR does not understand: \emph{The
control-flow graph 
lacks proper edges to indicate the data flow in the presence of
continuations.}
The first step is to identify exactly what the problem is, which
Christian will do.
\who{Christian.}
\item \b C
Add a target-specific map for hardware registers, and ensure that the
front end rejects named hardware registers that do not appear in the
map.
Might require changes in structure of front end to make it aware of
target, or the front end might accumulate a set of such registers and
have the checking done as an intermediate step.
N.B.~Christian recommends that exposing the target to the front end
will solve this and some other problems.
\who{Christian.}
\item \b C
Ensure that the front end rejects a named calling convention that is
not known to the back end.
\who{Christian.}
\item \b A
Targets provide code that spills and reloads registers. This code must
respect the machine invariant, but currently does not. Machine specific
spilling and reloading code must be defined. 
\who{John or Norman.}

(This code might be tied to
the question how small values of 8 or 16 bit widths are handled. Are
these values ever spilled?)
\item \b {B-}
Support \texttt{also cuts to k} on \texttt{cut to}. 
Here's an example use:
\begin{verbatim}
import put_ready, get_ready;
yield () {
  bits32 r;
  put_ready(k);
  r = get_ready();
  cut to r() also cuts to k;
  continuation k:
  return;
}
\end{verbatim}
\who{Christian}
\item \b{B-}
The \emph{semantics of the assembler interface} needs to be defined
clearly. The problem stems for an anticipated use for both textual
assemblers and binary emitters. Especially the role of symbols is not
clear and should be explained with respect to the better known textual
assemblers. 
The first step is to \emph{review the assembly interface}, which
should happen after the redesign of the flow graph.
\who{Norman.}
\item
In the assembler interface, there is a problem with the functions used
to emit initialized data---some ambiguity about how they are supposed
to be used.
Christian will identify the problem more precisely so it can be fixed.
\who{Christian and Norman.}
\item\b C
The semantics of \texttt{fcmp} should be specified precisely.
In particular, it is not known how the results map to two-bit values.
If the answer is target-specific, then the \texttt{Target} module should be
adapted accordingly.
This item has relatively low priority because the new floating-point
comparisons are more convenient than \texttt{fcmp}.
\who{?????}
\item \b{B+}
Alter the \texttt{lcc} back end to use the new floating-point
comparisons.
\who{????}
\item
The function \path|proc| that controls the translation of a procedure
in module \path|Ast3ir| must be broken up and cleaned up. 
The proper time to do this is \emph{after} the flow-graph revision.
\who{Christian}
\item\b{A-}
The treatment of the \texttt{*} operator in {\PAL} source code must be
corrected.
For purposes of type checking, 
this operator can be treated as a single polymorphic operator of that takes
two operands of width~$n$ and returns a result also of width~$n$.
For purposes of translation, however, the front end must generate a
composition of operators: first apply \path|mulu| to produce a result
of size~$2n$, then apply \path|lobits|$n$ to the result.\footnote
{We need to check the manual to see whether \texttt{*}~abbreviates a
signed or an unsigned multiply.}

For some reason, it appears to be difficult to handle this problem in
the front end.  
In that case, it will be reasonable to invent a new RTL operator, a
truncating multiply, perhaps to be called \texttt{mulu\_trunc}.
\who{Christian.}
\item \b A
The algorithm that linearizes the control-flow graph does not work
properly. It fails to detect the difference between nodes that fall into
a label and those that jump to a label. A probable fix to the problem is
always to jump to a label and never to fall into it. When the design of
the {\cfg} is reconsidered, this problem should be addressed, too.
\textit{I have worked around the problem by never falling into a label,
but using an explicit \path|goto|. This revealed some more problems with
the {\cfg} design: it is neccessary to create new nodes inside the
{\cfg} and thus no target-specific knowledge should be needed. I had to
give up the polymorphism of the {\cfg} which wasn't working anyway. --CL
(Tue May  7 16:44:58 EDT 2002)}.
\item \b C
We need support for IEEE 754 floating-point literals.
Mapping a literal to a bit vector is quite complicated and involved
(see papers from PLDI'96), but there should be reliable C~code on the
net to solve this problem, so that all we have to do is integrate it
into the compiler.
David Gay or Will Clinger may know where to find such code.
\who{Christian.}
\item \b C
We need a policy decision about whether to support floating-point
operations at compile time.
Someone will have to investigate to see if there is suitable ``soft
floating-point'' code available.

In the process, we should remove the suspect code from  \path|rtl/unit64p.c|.
This code provides casting of 64 bit floating point
values to 64 bit integer values, but it is platform
sensitive and maybe the wrong way.
\who{Christian.}
\item \b D
Implement support for string literals (low priority).
In order to support Unicode, we might need changes to the {\PAL}
language definition.
\item \b D
Identify a principled way to reject programs that a particular back
end cannot successfully compile.
Reasons for rejection might include use of {\PAL} variables or
operators at unsupported widths.

This problem is more important than a priority of \textbf{D} may
suggest, but it is a far-future item.
We will have a much better idea how to handle this problem after we
have built a few back ends.
\item\b{B-}
Ensure that \emph{global-variable declarations are consistent} across
modules. 
When multiple compilation units share their global variables one of them
must define them, while the other import them on an assembler level. No
mechanism exists to mark the ``main'' compilation unit that defines all
global variables.

N.B.~There is already code to compute a suitable MD5 signature.
Primarily what needs to be done is arrange suitable import/export and
implement a command-line option that designates the ``main''
compilation unit.
\item\b D
Implement \texttt{switch}.
\item\b C
Export substitution of late compile-time constants and constant folding
as finalizing step of a procedure translation to Lua. Currently these
steps are hard-coded into the \path|freeze| function in module
\path|Ast3IR|.
\end{enumerate}






Expression evaluation needs to be unified. The correct way is to
translate an expression into an {\rtl} expressions and then to evaluate
it. Currently the constant evaluation phase of the \path|Elab| module
evaluates constant expressions directly. See also the reasons for this
in module \path|Elab.Const.Eval|:
\begin{quote}\it
    Several reasons: We would need a translation from {\AST} to
    \path|Rtl.env|. The existing one in \path|ast3ir| is written against
    the \path|Fenv.Clean| interface, but we need one written against the
    \path|Fenv.Dirty| interface. The translation of an application needs
    the widths of its arguments. Therefore type-checking using
    \path|Expcheck| would be done twice: once before translation and then
    again during translation. Detailed error messages are simpler to
    generate with the existing approach. 
\end{quote}




\paragraph{Old problems that have been solved}
After an {\rtl} has left the register allocator, it may still contain
late compile-time constants. After these are substituted by bit vectors,
the {\rtl} must undergo a constant folding process to make sure it
respects the machine invariant. Such a constant folding routine
currently does not exist.  Since the front end also needs to evaluate
compile-time and link-time {\rtl} expression, one general {\rtl}
evaluation mechanism would be appropriate. The current module
\path|rtleval| can only handle compile-time and link-time {\rtl}s. 
\textit{I hope this problem is solved in \path|Rtleval2|. --CL}



\paragraph{Items whose status is unclear}
It is not clear whether this is a problem that needs immediate
attention or a suggestion for refactoring.
\begin{itemize} 
\item Relocatable addresses have undergone several changes in their
representation.  There are still two slightly different representations
used in the compiler that wait to be unified: \path|Reloc| and one in
\path|Sledlib|.  
\end{itemize}


\paragraph{Obsolete items}
\nobreak
\begin{itemize}
\item Integration of machine generated code. Basically it does not work
because machine generated code is not flexible enough to handle
temporaries and late compile-time constants. Adjusting the machine
generated code manually is time-consuming and error prone. Machine
generated code provides {\rtl} constructors for machine instructions, a
recognizer for machine instructions in {\rtl} form, and an emitter of
assembly code.
\emph{This item will be delayed until after we have a working
compiler.}
\end{itemize}



\paragraph{Suggestions for refactoring}

The main overall problem is complexity. The development of the compiler
will benefit from trying to cut down the number of modules,
abstractions, and moving parts in general. 
\begin{itemize}
\item
There is some confusion in the way {\PAL} primitive operators are
translated to {\rtl} operators. {\PAL} operators are monomorphic in
their result type, whereas {\rtl} operators are not. A polymorphic
{\rtl} operator takes an additional argument compared to its monomorphic
{\PAL} counter part. This argument should be supplied first to take
advantage of currying: the a polymorphic {\rtl} operator supplied with
its first argument implements the monomorphic {\PAL} operator. Currently
the additional argument is the last argument. \textit{I hope this
problem is solved in \path|Rtlop|. The above explanation of operator
specialization is not entirely correct. The specializing parameter is
not a formal argument of a primitive, but part of its \emph{width list}.
--CL}
\item
The semantics of assembler symbols have leaked into the front end, the
fat environment, and the {\cfg}. Symbols are handed out by an assembler
which means, a reference to the current assembler must be available to
create new symbols. Because the {\cfg} does not include such a
reference, it is difficult to create \path|goto| nodes (during {\cfg}
manipulation) outside the \path|Ast3IR| module.  This is related both to
the point above, and the unclear semantics of the assembler interface,
and symbols in particular.  I would like to avoid dealing with target
specific symbols as much as possible and would like to delegate their
creation into the last phases of compilation. Upon further reflection I
now believe that the \emph{construction} of a {\cfg} should not require
any machine-specific knowledge to allow for the easy implementation of
{\cfg} transformations.
\end{itemize}



% ------------------------------------------------------------------ 
\end{document}
% ------------------------------------------------------------------ 

